\chapter{Challanges}


In some cases it might be hard to evaluate the output, e.g. in a translation problem where there might be multiple correct outputs. In our case we sort of have the same problem in that there might be multiple items that the user would click on if presented to him.

So the recommender systems score on the test set might not be a very good indicator of how good a recommender system is. Can be hard to tell whether a system is truly strong or if it only overfits the performance metrics.\\

Much of the recent improvement and success of RNNs come from the exploration of new architectures. Therefore, much of the work in making a good recommender system with RNN might be to find a good architecture. Hopefully we can use lessons and knowledge from similar problems where RNNs have been used with promising results