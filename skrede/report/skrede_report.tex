\documentclass[a4paper, twoside, titlepage, 12pt]{report}

\usepackage[T1]{fontenc}					% norsk tegnsett (æøå)
\usepackage[utf8]{inputenc}					% norsk tegnsett
\usepackage{geometry}						% anbefalt pakke for å styre marger.

\usepackage{amsmath,amsfonts,amssymb} 		% matematikksymboler
\usepackage{amsthm}                   		% for å lage teoremer og lignende.
\usepackage{graphicx}                 		% inkludering av grafikk
\usepackage{subfig}                   		% hvis du vil kunne ha flere figurer inni en figur
\usepackage{listings}     		            % main aim is to include the source code of any programming language within your document

\usepackage[nottoc]{tocbibind} 				% Includes references in table of contents
\usepackage{acronym}						% This pack­age en­sures that all acronyms used in the text are spelled out in full at least once.
\usepackage{hyperref}           		    % Lager hyperlinker i evt. pdf-dokument
\usepackage{url}							%

\usepackage{gensymb}						% to display degrees (celsius, ..)

\usepackage{xcolor}
\newcommand\TODO[1]{\textcolor{red}{#1}}			% Add red TODOs in the paper
\newcommand{\source}[1]{\caption*{\footnotesize{Source: {#1}}} }	% Used to cite images

% Title Page
\title{Title}
\author{Ole Steinar Lillestøl Skrede}


% ==============================================
% ==============================================
\begin{document}
	
%!TEX root = template_report.tex
% !TEX spellcheck = en-US

%This is the Titlepage
%%=========================================
\thispagestyle{empty}
\includegraphics[scale=1.1]{fig/logos/NTNU}
\mbox{}\\[6pc]
\begin{center}
\Huge{Recurrent Neural Network for Recommendation and User Representation}\\[2pc]

\Large{Ole Steinar Lillestøl Skrede}\\[1pc]
\large{November 2016}\\[2pc]

Department of Computer and Information Science\\
Norwegian University of Science and Technology
\end{center}
\vfill

\noindent Supervisor: Massimiliano Ruocco



% Preface
%%=========================================
\pagenumbering{roman}
\begin{abstract}
	Many websites face the task of providing good recommendations to users which they know little about. E.g. an e-commerce site where many users are not logged in, or most users are new to the site. Recommender models based on having user profiles, struggle in this setting. The solution is often to use item-to-item models, where recommendations are made based on similar items. This approach does not utilize the full sequence of actions in a user session. Recurrent neural networks (RNNs) have the capability to perform predictions based on sequences. Recently, research has been done on applying RNNs in the session-based recommender setting. The results have been promising, RNNs have been shown to outperform state of the art models. We explore research on RNNs as recommender systems, and build our own basic RNN recommender model, which we will use for further research. A simple RNN model can perform well enough that it should be considered for practical applications, and several ways of further improving the performance has been studied. Apart from data augmentation techniques and customized training approaches, supplying the RNN model with additional information can improve performance. External contextual information such as time and weather, time between user actions in the session, and additional item descriptions such as text and images, can all be used to improve predictions.
\end{abstract}
% Table of contents
\tableofcontents \clearpage
%\listoffigures \clearpage

% Content
%%=========================================
\setcounter{page}{0}
\pagenumbering{arabic}
\chapter{Introduction}
Give short introduction about what this report will discuss (rnn), touch very briefly upon what the reader should have in mind and can expect from the report. Explain what will be discussed in the different chapters.

Touch very briefly upon what rnn is, so that the reader understands enough to understand the next sections, before we get to background. (Or should background come before motivation and objectives of work?)


\section{Motivation}
Why are RNNs interesting?
- easier feature extraction/engineering (zalando blog)
- the model fits very well with problems involving sequences. can model sequences, has memory
- has achieved very promising and state of the are results 
- has recently become interesting because of lstm, gru which helps with vanishing and exploding gradients, pluss powerful hardware
- does not suffer from cold-start problem. can apply on unknown users
- can easily work with sequences of different length

\section{Objectives of the work}
- look at work that has been done, state of the art
- look at how results can be improved when predicting sequences
- specifically interested in situations that apply to online users where we might have no prior knowledge of the user

\section{Background}
- explain the problem domain
- explain rnn
- 
\chapter{State of the Art}
mmmmmmh
\chapter{RNN for session-based recommendation}
In this chapter we explain the work we did on implementing a RNN for session-based recommendations. We used the Tensorflow \cite{tensorflow2015-whitepaper} library, and our code is available here \cite{skrede:code}.

\section{Goal}
From the papers we have looked at in Chapter \ref{chp:sota}, it is clear that a RNN can perform really well as a session-based recommender. It is also clear that there are possibilities to improve performance through different techniques and architectures. To be able to do our own experimentation and exploration in the domain we needed to implement a model of our own. We decided to use Tensorflow to implement a model heavily inspired by the model by Hidasi et al. in \cite{DBLP:journals/corr/HidasiKBT15}. This would allow us to familiarize ourselves with the software library and give us a model that could be used for further experimentation. We wanted our model to achieve similar performance to the one created by Hidasi et al.

\section{Implementation}
We have described the model created by Hidasi et al. in Section \ref{sec:hidasi-sess-based-rnn} and in Figure \ref{fig:gru4rec-network}. They used Theano for their implementation, the code is available at GitHub \cite{hidasi:code}. In this section we describe our implementation. Since our model is similar to the one by Hidasi et al., the reader is referred to \ref{sec:hidasi-sess-based-rnn} for more details. Here we focus on differences between the two models, and only briefly describe the similar parts. Our model is illustrated in Figure \ref{fig:skrede-rnn}

\begin{figure}[htp]
	\centering
	\includegraphics[width=1.0\textwidth]{fig/skrede-rnn.png}
	\caption{Illustration of the model we implemented, inspired by the model from \cite{DBLP:journals/corr/HidasiKBT15}.}
	\label{fig:skrede-rnn}
\end{figure}

The input is a one-hot representation of the clicked item in each sequence, which is sent into a GRU layer with 100 units. Between each sequence, the hidden state of the GRU layer is reset. Dropout is applied to the output from the GRU layer, with a dropout rate of 50\%. Dropout is not applied during testing. Then the output is sent through a feedforward layer and outputs scores for all items. Softmax cross-entropy is used on the output, with the one-hot representation of the next click as the label, to compute the loss. Adam \cite{DBLP:journals/corr/KingmaB14} is used for training, with a learning rate of 0.001. During testing, the indexes of the top K scores are retrieved as the models predictions.

Some of our choices and parameters are chosen somewhat random based on what we know worked well in similar models. We wanted to test with different parameters and solutions to optimize our model, unfortunately we were stuck for a long time, trying to get the model to run at a feasible speed. Thus, some of the parameters can probably be optimized.

Cross-entropy for calculating loss was chosen because Tensorflow already has support for it. We added dropout as we wanted to see what effect it would have on performance (both accuracy and runtime). All sequences were padded to the same length, for example length 10. This was because Tensorflow requires it. The padded clicks are outputted as all-zero vectors from the GRU layer. We wanted to experiment with adding bias in the feedforward layer, but because we only recently were able to use masking to ignore the padded all-zero vectors, bias has not been used. The problem is that the bias would have been applied to the padded vectors, which would affect training. Training was done in mini-batches of size 200, and the examples in each mini-batch were sampled randomly from the dataset.

We tested the effect of using a feedforward layer by comparing the model with and without that layer. We only tested this on our smallest dataset, but the model was both faster and more accurate when a feedforward layer was used. 
%explain the paper what i have tried to implement
%what did i want to achieve 
%core of what i implemented
\chapter{Experiments and results}

\section{Experiments}

\section{Results}

\section{Discussion}



%explain the experimental setting
%	explain the dataset that I have used
%		explain everything about the two datasets
%	talk about evaluation metrics 
%		how do they fit the recsys problem
%	which kind of testing
%	description of the experiment
	
%	results
	
%	discussion
\chapter{Conclusion and further work}

\section{Conclusion}


%summarize what has been done and what has been the challanges

\section{Further work}


%what is the future work
%have only looked at actions so fat
%we want to look at feeding in contextual data

%representation of the user

%- user representation with rnn (as the rnn works through a sequence, it stores a representation of user preferences, this could be used to represent the user. That is, the rnn creates a session representation and by combining these session representations for an individual user, it should be possible to represent the user.)
%- Use contextual information to improve the rnn recommender system

% Appendix
%%=========================================
\appendix
\chapter{List of Acronyms}

%\addcontentsline{toc}{section}{List of Acronyms}

% Keep these sorted alphabetically for extra readability

\begin{acronym}
    \acro{RNN}{Recurrent Neural Network}
\end{acronym}


%\include{appendix-b}

\bibliographystyle{unsrt}
\bibliography{references}{}

\end{document}
% ==============================================
% ==============================================



% ==============================================
% =========== LATEX EXAMPLES ===================
% ==============================================

% =========== Code - listing ===================
\begin{lstlisting}
Put your code here.
\end{lstlisting}

\lstinputlisting[language=Python, firstline=37, lastline=45]{source_filename.py}

% =========== TODO =============================
\TODO{What to do}